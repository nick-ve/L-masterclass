\Transcb{yellow}{blue}{Vooruitzichten}
\onecolumn
\begin{itemize}
\item \colorbox{yellow}{IceCube : 's Werelds grootste neutrino observatorium op de Zuidpool}
\item[] De volledige IceCube detector is sinds december 2010 in bedrijf
\item[] IceCube sensoren werken naar behoren (Maanschaduw, hemelkaart)
\item \colorbox{yellow}{Zeer gedetailleerd onderzoek van de "neutrino hemel"}
\item[] Valt in tijd mooi samen met satelliet waarnemingen (Swift, Fermi)
\item[] $\rightarrow$ Perfect voor GRB onderzoek
\item \colorbox{yellow}{Wereldprimeur : Kosmische hoog-energetische neutrino's ontdekt}
\item[] \begin{center}{\blue De geboorte van Neutrino Astronomie}\end{center}
\item {\red Onderzoek aan het IIHE :}
\item[] Nieuwe methode voor detectie van GRB neutrino's
\item[] Onderzoek naar neutrino productie in zonnevlammen
\item[] Nieuw idee voor neutrino detectie van actieve melkwegkernen 
\end{itemize}
%
\begin{center}
\colorbox{yellow}{Er breken zeer interessante tijden aan voor onze Astrodeeltjes Fysica !}
\end{center}
